\textbf{Martin:} Several months prior to start of this project, I became a bit interested in machine learning(ML). I thought that getting some extra knowledge in seemingly tiny area of default prediction models could help me with general understanding of the ML field and maybe even with completing some of my Coursera course tasks. What really surprised me though, is that fault prediction is not just a tiny area on the intersection of computer science, statistics and machine learning, but it is rather an already big and still growing area of research. Going into the project with the mindset of "underestimating" this field, I experienced some struggles with designing an experiment (because of many possible options). Once I got frustrated, I just started to do some measurements hoping that it will lead me to something. Well, it led me to later realization that the experiment(measurements) didn't make much sense. And it's not the first time such scenario happened to me. Except for all other gained knowledge, I've once again learned that \textbf{"You better measure one more time before you do the cut"} (typical Slovakian saying my grandmother used to tell me all the time) does really make sense.\\\\
\textbf{Alberte:} When I first heard about Software Fault Prediction, I thought it was a waste of time since I couldn't imagine that a model could be found that actually provided good results. So beginning this project my expectations was that we would get really bad results and that we would have troubles finding good documentation and papers on the subject. And it turned out that I was completely wrong. I still think that it is important that each company and project takes the time to figure out if doing fault prediction makes sense for them, but see now, how much it can matter for a project and company. I have also been very positively surpriced that the results of the predictors are so good, in general. \\ What has been most difficult for me in this project has been to decide on the type of experiments we wanted to do, since there are so many possibilities and every one of them could be interesting to do. But I think it is a very important learning point to learn to choose what is most important and start with that, and I do think this is what we ended up doing. 