\textbf{Martin:} Several months prior to start of this project, I became a bit interested in machine learning(ML). I thought that getting some extra knowledge in seemingly tiny area of default prediction models could help me with general understanding of the ML field and maybe even with completing some of my Coursera course tasks. What really surprised me though, is that fault prediction is not just a tiny area on the intersection of computer science, statistics and machine learning, but it is rather an already big and still growing area of research. Going into the project with the mindset of "underestimating" this field, I experienced some struggles with designing an experiment (because of many possible options). Once I got frustrated, I just started to do some measurements hoping that it will lead me to something. Well, it led me to later realization that the experiment(measurements) didn't make much sense. And it's not the first time such scenario happened to me. Except for all other gained knowledge, I've once again learned that \textbf{"You better measure one more time before you do the cut"} (typical Slovakian saying my grandmother used to tell me all the time) does really make sense.