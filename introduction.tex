Software Fault Prediction(SFP) is one of the most active research areas in software engineering. In spite of diligent planning, documentation, and proper process adherence in software development, occurrences of defects are inevitable. Finding and fixing defects costs companies around the world huge amounts of money. Therefore, any automated help in reliably predicting where faults are, and thereby focusing the efforts of testers, has a significant impact on the cost of production and maintenance of software. Various regression techniques, and recently also machine learning algorithms, have been utilized to provide better insight into software repositories and help developers as well as testers to invest their time at work more effectively. Some professionels voices a critical concern about the importance and rentability of creating a defect prediction models, as part of software development projects. However, there is no doubt that such a model, if implemented correctly, represents an additional tool to increase work effectiveness as well as software reliability.